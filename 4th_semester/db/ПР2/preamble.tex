\documentclass[14pt]{report} % Размер шрифта 14 pt
\usepackage{fontspec} % Для использования системных шрифтов
\usepackage{geometry} % отступы документа
\usepackage[utf8]{inputenc} % Для поддержки UTF-8
\usepackage[russian]{babel} % Для русского языка
\usepackage{datetime} % Пакет для работы с датами
\usepackage{tabularx} % Пакет для работы с таблицами
\usepackage{titlesec} % Для смены стилей разделов
\usepackage{indentfirst} % Для отступа первой строки в первом абзаце
\usepackage{enumitem} % Продвинутая работа со списками
\usepackage{fancyhdr} % Нумерация
\usepackage{setspace} % Подключаем пакет для управления межстрочным интервалом
\usepackage{caption} % Работа с изображениями
\usepackage{graphicx} % Работа с изображениями
\usepackage{booktabs} % Для тонких линий
\usepackage{float} % поведение изображения на странице


\geometry{top=20mm, bottom=20mm, left=30mm, right=10mm}

\setmainfont{Times New Roman} % Устанавливаем Times New Roman как основной шрифт
% Определяем новый стиль шрифта для подписей
\DeclareCaptionFont{myfont}{\fontsize{14}{14}\selectfont}

\linespread{1}
\singlespacing % Устанавливаем одинарный интервал для всего документа
% Глобальный отступ первой строки
\setlength{\parindent}{1.25cm} % Отступ первой строки для всех абзацев

\renewcommand{\headrulewidth}{0pt} % Убирает верхнюю линию колонтитула

% Настройка списков
\setlist{
  nosep, % Убирает лишние отступы и интервалы
  topsep=0pt, % Отступ сверху
  partopsep=0pt, % Дополнительный отступ сверху
  itemsep=0pt, % Отступ между элементами списка
  parsep=0pt, % Отступ между абзацами внутри элементов списка
  leftmargin=* % Выравнивание списка по левому краю
}

\titleformat{\part}
  {\normalfont\fontsize{14}{14}\bfseries\centering} % Форматирование
  {} % Метка (не используется, так как раздел ненумерованный)
  {0pt} % Отступ перед заголовком
  {} % Код перед заголовком

% Настройка всех уровней секций
\titleformat{\chapter}
  {\normalfont\fontsize{14}{14}\bfseries\centering} % Форматирование
  {} % Метка (не используется, так как раздел ненумерованный)
  {0pt} % Отступ перед заголовком
  {} % Код перед заголовком

\titleformat{\section}
  {\fontsize{14}{14}\bfseries} % Форматирование
  {\hspace{1.25cm}\thesection} % Метка (не используется, так как раздел ненумерованный)
  {\labelsep} % Отступ перед заголовком
  {} % Код перед заголовком

\titleformat{\subsection}
  {\normalfont\fontsize{14}{14}\bfseries} % Форматирование
  {\thesubsection} % Метка (номер подраздела)
  {\labelsep} % Отступ перед заголовком
  {} % Код перед заголовком

\titleformat{\subsubsection}
  {\normalfont\fontsize{14}{14}\bfseries} % Форматирование
  {\thesubsubsection} % Метка (номер подподраздела)
  {\labelsep} % Отступ перед заголовком
  {} % Код перед заголовком

\titleformat{\paragraph}
  {\normalfont\fontsize{14}{14}\bfseries} % Форматирование
  {\theparagraph} % Метка (номер параграфа)
  {\labelsep} % Отступ перед заголовком
  {} % Код перед заголовком

\titleformat{\subparagraph}
  {\normalfont\fontsize{14}{14}\bfseries} % Форматирование
  {\thesubparagraph} % Метка (номер подпараграфа)
  {\labelsep} % Отступ перед заголовком
  {} % Код перед заголовком

% Настройка отступов вокруг заголовков
\titlespacing*{\chapter}{0pt}{0pt}{0pt}
\titlespacing*{\section}{0pt}{14pt}{14pt} % Отступы для \section
\titlespacing*{\subsection}{0pt}{0pt}{1em} % Отступы для \subsection
\titlespacing*{\subsubsection}{0pt}{0pt}{1em} % Отступы для \subsubsection
\titlespacing*{\paragraph}{0pt}{0pt}{1em} % Отступы для \paragraph
\titlespacing*{\subparagraph}{0pt}{0pt}{1em} % Отступы для \subparagraph

% Настройка нумерации
\pagestyle{fancy} % Включаем стиль fancy для колонтитулов
\fancyhf{} % Очищаем колонтитулы
\fancyfoot[C]{\thepage} % Добавляем номер страницы по центру внизу

\renewcommand{\thesection}{\arabic{section}} % Начинаем нумерацию секций с 1 в каждом разделе \chapter
\counterwithout{figure}{section} % Отвязываем \chapter от \section

% Настраиваем формат подписи с длинным тире
\captionsetup[figure]{
  labelsep=endash, % Используем длинное тире (—)
  format=plain,    % Простой формат (без дополнительных стилей)
  justification=centering, % Выравнивание по центру
  skip=14pt, % Отступ между изображением и подписью
  font=myfont  % Размер шрифта подписи (14pt)
}

% Настройка изображений
\graphicspath{ {Images/} } % Путь до изображений
\setlength{\intextsep}{0pt} % Отступ вокруг плавающего объекта
\setlength{\topsep}{0pt} % Отступ сверху и снизу плавающего объекта
